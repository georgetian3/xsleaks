\subsection{\code{postMessage}}

\code{window.postMessage()} 实现了不同 \code{Window} 对象之间的跨站点通讯,比如一个网页和它创建的弹窗或它之中的 \code{iframe} \cite{pm}。\code{postMessage()} 所发出的 \code{MessageEvent} 这个事件可以泄漏信息,比如如果一个网站在用户名存在的情况下发出 \code{MessageEvent},则可以判定一个用户的用户名。此外,\code{MessageEvent} 中也可以含有敏感信息。

防止攻击者滥用此漏洞目前是网页开发者的责任。开发者需要保证网页在不同的(登录、授权等等)状态下,\code{postMessage} 的行为是一致的。此外,应该合理设定 \code{postMessage} 的  \code{targetOrigin} 参数,使得只有目标站点可以读取 \code{MessageEvent} 中的信息。
